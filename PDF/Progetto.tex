
\documentclass[paper=a4, fontsize=12 pt]{scrartcl} % A4 paper and 11pt font size
\usepackage[top=90pt,bottom=100pt,left=66pt,right=66pt]{geometry}

\raggedbottom

\usepackage{xcolor, listings}
\usepackage{textcomp}
\usepackage{color}

\definecolor{codegreen}{rgb}{0,0.6,0}
\definecolor{codegray}{rgb}{0.5,0.5,0.5}
\definecolor{codepurple}{HTML}{C42043}
\definecolor{backcolour}{HTML}{F2F2F2}
\definecolor{bookColor}{cmyk}{0,0,0,0.90}  
\color{bookColor}

\lstset{upquote=true}

\lstdefinestyle{mystyle}{
	backgroundcolor=\color{backcolour},   
	commentstyle=\color{codegreen},
	keywordstyle=\color{codepurple},
	numberstyle=\numberstyle,
	stringstyle=\color{codepurple},
	basicstyle=\footnotesize\ttfamily,
	breakatwhitespace=false,
	breaklines=true,
	captionpos=b,
	keepspaces=true,
	numbers=left,
	numbersep=10pt,
	showspaces=false,
	showstringspaces=false,
	showtabs=false,
}
\lstset{style=mystyle}

\newcommand\numberstyle[1]{%
	\footnotesize
	\color{codegray}%
	\ttfamily
	\ifnum#1<10 0\fi#1 |%
}


\usepackage{tabularx}
\usepackage[]{graphicx}
\usepackage[]{color}
\usepackage{color, soul}
\usepackage{alltt}
\usepackage[T1]{fontenc}
\usepackage[utf8]{inputenc}
\usepackage{ragged2e}
\usepackage{amsmath}
\usepackage{multicol}
\usepackage{multirow}
\usepackage{environ}
\usepackage{tikz}
\usetikzlibrary{matrix,calc, arrows, shapes.gates.logic.US,}
\usetikzlibrary{positioning}
\usepackage{graphicx}
\usepackage{bm}
\usepackage{mathtools}
\usepackage{pst-node}
\usepackage{circuitikz}
\usepackage{parskip}
\usepackage[normalem]{ulem}
\usepackage{cancel}
\usepackage[english, italian]{babel}
\selectlanguage{italian}
\def\checkmark{\tikz\fill[scale=0.4](0,.35) -- (.25,0) -- (1,.7) -- (.25,.15) -- cycle;} 

\usepackage{float}

\usepackage{hyperref}
\hypersetup{
	colorlinks=true,
	linkcolor=blue,
	filecolor=magenta,      
	urlcolor=cyan,
	pdftitle={Overleaf Example},
	pdfpagemode=FullScreen,
}

\graphicspath{ {Figures/} }



%----------------------------------------------------------------------------------------
%	TITLE SECTION
%----------------------------------------------------------------------------------------

\newcommand{\horrule}[1]{\rule{\linewidth}{#1}} % Create horizontal rule command with 1 argument of height

\title{
	\normalfont \normalsize
	\textsc{Università degli Studi di Trieste - Dipartimento di Ingegneria
		e Architettura} \\ [25pt]
	\horrule{0.5pt} \\[0.4cm]
	\huge Progetto - Basi di Dati\\
	\emph{"Olimpiadi"}
	\horrule{2pt} \\[0.5cm]
}

\author{Lorenzo Martin Diaz Avalos \\ IN0501010}

\date{\normalsize Anno Accademico 2022-2023}

\begin{document}
	\maketitle
	\tableofcontents
	\newpage
	
	\newpage
	
	\section{Progettazione}
	
	
	Ogni due anni si organizza l' edizione delle olimpiadi e si vuole realizzare un semplice database per gestire i dati.\\\\
	Le olimpiadi si svolgono in un determinato periodo identificato da una data di apertura e chiusura ed è presente una cerimonia di inizio e fine edizione, inoltre hanno un simbolo rappresentativo, identificato tramite la mascotte.\\
	Le edizioni delle olimpiadi sono composte da diversi giochi olimpici, identificati dalla disciplina, che hanno una piccola descrizione del gioco stesso. Ad ogni gioco possono competere più atleti, anche rispetto a più edizioni ed ogni atleta riceve una posizione rispetto alla gara a cui compete (può succedere che più atleti effettuino lo stesso risultato e che quindi abbiano la stessa posizione).\\\\
	Di ogni atleta si vuole conoscere il nome, cognome, la data di nascita e la nazione da cui proviene, inoltre devono aver compiuto almeno il 18-mo anno di età prima della data di inizio dell' edizione a cui vogliono partecipare.\\\\
	Le olimpiadi hanno luogo in una città, identificata dal loro nome e le città si trovano in un paese. Ogni città mette a disposizione un investimento in denaro per l' organizzazione dell' olimpiade. Le città sono divise per zone, in modo da riuscire a coprire tutta la città e ad assegnare i vari stadi in ogni zona.\\
	Degli stadi si vuole sapere il nome e la capienza massima.\\\\
	Le postazioni dello stadio possono essere prenotate attraverso i biglietti, che hanno un loro codice identificativo, una loro postazione, che non può superare la capienza dello stadio ed il nominativo della persona che acquista il biglietto, quindi nome, cognome ed età. Solo i maggiorenni possono acquistare il biglietto, quindi gli acquirenti possono acquistare più di un biglietto.
	
	\newpage
	
	\section{Operazioni}
	
	\begin{enumerate}
		\item Si vuole ottenere la classifica delle nazioni con più atleti sul podio rispetto ad una data edizione (una volta ogni due anni)
		\item Si vogliono ottenere gli investimenti ed il numero di stadi messi a disposizione da ogni città per ogni edizione (una volta ogni due anni)
		\item Si vogliono ottenere il numero di biglietti rimanenti di tutti gli stadi di una data edizione (20 volte al mese)
		\item Si vogliono ottenere gli atleti più premiati (sul podio) rispetto a tutte le edizioni (una volta ogni due anni)
	\end{enumerate}
	
	
	\newpage
	
	\section{Diagramma Entity - Relationship}
	
	\includegraphics[width=25cm, height=20cm]{Figures/Giochi Olimpici.pdf}
	
	In BLU ci sono le entità, in ARANCIONE le relazioni, in VERDE gli attributi ed in VERDE SCURO gli identificatori delle entità.
	
	
	
	\newpage
	
	\section{Dizionario dei dati}
	\subsection{Dizionario delle entità}
	
	
	\begin{table}[H]
		\centering
		\begin{tabular}{ ||p{3.5cm}|p{3.5cm}|p{3.5cm}|p{3.5cm}||  }
			\hline
			\hline
			\textbf{Entità} & \textbf{Descrizione} & \textbf{Attributi} & \textbf{Identificatore}\\
			
			\hline
			\hline
			
			Olimpiade   & Edizione dell' olimpiade   & edizione, anno, mascotte& edizione \\
			\hline
			Cerimonia &   Celebrazione di apertura e chiusura  & codice, data di apertura, data di chisura   & codice \\
			\hline
			Atleta & Atleti partecipanti all' olimpiade & ID, nome, cognome, nazione, data di nascita & ID \\
			\hline
			Gioco    & Gara di uno sport specifico & disciplina, descrizione & disciplina \\
			\hline
			Città &   Città in cui si svolge l' olimpiade  & nome, paese, investimento & nome \\
			\hline
			Zona & Coperture della città & nome, copertura & nome \\
			\hline
			Stadio & Impianti a disposizione dell' olimpiade  & nome, capienza & nome \\
			\hline
			Biglietto& Biglietto  & codice, postazione, nominativo   & codice \\
			\hline
			\hline
		\end{tabular}
	\end{table}
	
	\subsection{Dizionario delle relazioni}
	
	\begin{table}[H]
		\begin{tabular}{ ||p{3.6cm}|p{3.6cm}|p{3.6cm}|p{3.6cm}||  }
			\hline
			\hline
			\textbf{Relazione} & \textbf{Descrizione} & \textbf{Componenti} & \textbf{Attributi}\\
			
			\hline
			\hline
			
			Celebrazione & celebrazione di un' edizione dell' olimpiade & codice cerimonia, edizione olimpiade & \hspace{1.5cm}- \\
			\hline
			Composizione & gioco di una specifica edizione dell' olimpiade & edizione olimpiade, disciplina & \hspace{1.5cm} - \\
			\hline
			Competizione & Gara a cui partecipano gli atleti & ID, disciplina, edizione olimpiade & \hspace{3cm} posizionamento \\
			\hline
			Luogo & luogo in cui si svolge una specifica edizione dell' olimpiade & edizione olimpiade, nome città & \hspace{1.5cm} - \\
			\hline
			Sede & luoghi di una specifica città  & nome città, nome zona & \hspace{1.5cm} - \\
			\hline
			Impianto & impianti a disposzione di una specifica zona & nome zona, nome stadio & \hspace{1.5cm} - \\
			\hline
			Prenotazione & prenotazione del biglietto per uno specifico stadio  & nome stadio, codice biglietto & \hspace{1.5cm} - \\
			\hline
			\hline
		\end{tabular}
	\end{table}
	
	
	\newpage
	
	
	\section{Vincoli non esprimibili graficamente}
	
	I vincoli non esprimbili graficamente sono:
	\begin{itemize}
		\item Gli atleti devono avere almeno 18 anni di età rispetto all' edizione dell' olimpiade a cui partecipano
		\item Il numero di biglietti venduti per uno stadio deve essere minore della capienza dello stesso stadio
		\item Il numero che rappresenta la postazione di un biglietto per uno stadio non può essere maggiore della capienza dello stesso stadio
		\item Il numero che rappresenta la postazione di un biglietto per uno stadio non può essere duplicato
		\item L' età di chi acquista un biglietto per un qualsiasi stadio deve essere almeno di 18 anni
		\item La data di apertura di un' edizione di un' olimpiade deve essere precedente alla data di chiusura della stessa edizione dell' olimpiade
		\item Il numero che rappresenta il posizionamento di un atleta deve essere positivo
		
	\end{itemize}
	
	\newpage
	
	\section{Tavola dei volumi}
	
	\begin{center}
		\begin{tabularx}{0.8\textwidth}{ 
				|| >{\centering\arraybackslash}X 
				>{\centering\arraybackslash}X 
				>{\centering\arraybackslash}X || }
			\hline
			\hline
			\textbf{Concetto} &  \textbf{Tipo} & \textbf{Volume} \\
			\hline
			\hline
			Olimpiade & E & 33 \\
			\hline
			Celebrazione  & R  & 33\\
			\hline
			Cerimonia  & E  & 33\\
			\hline
			Gioco Olimpico  & E  & 660 \\
			\hline
			Composizione  & R  & 660\\
			\hline
			Atleta  & E  & 40000\\
			\hline
			Competizione  & R  & 80000\\
			\hline
			Sede  & R  & 33\\
			\hline
			Città  & E  & 25\\
			\hline
			Luogo  & R  & 125\\
			\hline
			Zona  & E  & 125  \\
			\hline
			Impianto  & R  & 625\\
			\hline
			Stadio  & E  & 625 \\
			\hline
			Prenotazione  & R  & 6250000\\
			\hline
			Biglietto  & E  & 6250000 \\
			\hline
			\hline
		\end{tabularx}
		
	\end{center}
	
	
	
	\newpage
	\section{Analisi generale}
	
	
	Si può osservare che l' attributo "Nominativo" dell' entità "Biglietto" è un attributo multivalore, contiene altri attributi che, rispetto a questa scelta di modellazione, non sono attributi che riguardano l' entità "Biglietto", quindi si è deciso di creare una nuova entità "Spettatore" che conterrà i suoi attributi "Nome", "Cognome", "Età" e "CF" aggiungendo la relazione "Acquisto" che collega le due entità di modo che ogni spettatore possa acquistare più biglietti ed ogni biglietto è acquistato da un solo spettatore.\\\\
	Un' altra osservazione riguarda l' attributo "Anno" dell' entità "Olimpiade". Questo attributo si può calcolare direttamente attraverso l' attributo "Data Di Apertura" dell' entità "Cerimonia", perciò sarebbe un campo calcolato e non sarebbe rispettata la terza forma normale. Si è quindi deciso di toglierlo.\\\\
	Osserviamo poi che l' entità "Gioco Olimpico" è identificato tramite la disciplina e l' identificatore esterno "Olimpiade", questo perché un gioco olimpico può essere ripetuto su più edizioni. \\\\
	Per quanto riguarda la scelta degli identificatori primari si sono scelti in questo modo:
	\begin{itemize}
		\item Olimpiade: edizione dell' olimpiade
		\item Cerimonia: codice cerimonia
		\item Gioco Olimpico: disciplina e Olimpiade a cui riferisce
		\item Atleta: codice identificativo
		\item Città: nome della città
		\item Zona: nome della zona
		\item Stadio: nome dello stadio
		\item Biglietto: codice identificativo del biglietto
		\item Spettatore: codice fiscale dello spettatore\\
	\end{itemize}
	
	Ulteriore osservazione riguarda l' entità "Biglietto". Dato che l' attributo "Postazione" deve essere non nullo e numericamente diverso per tutti i dati rispetto ad uno specifico stadio, si poteva scegliere di usare l' attributo "Postazione" come chiave primaria insieme all' identificatore esterno attraverso la relazione "Prenotazione" ma si è comunque scelto di usare un codice come chiave primaria per rispettare più una visione reale nella quale un biglietto di uno stadio non è identificato dal numero della postazione ma da un codice univoco.\\\\
	Riguardo la progettazione fisica, ho deciso di non usare indici secondari dato che non ci sono attributi utilizzati con alta frequenza nelle ricerche.
	
	
	\newpage
	\section{Diagramma Entity - Relationship Ristrutturato}
	
	
	\includegraphics[width=25cm, height=20cm]{Figures/Giochi Olimpici Ristrutturato.pdf}
	
	
	\newpage
	
	
	
	\section{Schema Logico}
	
	\includegraphics[width=15cm, height=15cm]{Figures/giochi_olimpici.png}
	
	\begin{itemize}
		\item città (\underline{nome}, paese, investimento)
		\item cerimonia (\underline{codice}, dataApertura, dataChiusura)
		\item olimpiade (\underline{edizione}, mascotte, sede, cerimonia)
		\item zona (\underline{nome}, zonaCopertura, città)
		\item stadio (\underline{nome}, capienza, zona)
		\item spettatore (\underline{CF}, nome, cognome, età)
		\item gioco (\underline{disciplina}, \underline{edizione}, descrizione)
		\item atleta (\underline{ID}, nome, cognome, nazione, dataNascita)
		\item biglietto (\underline{codice}, postazione, stadio, spettatore)
		\item competizione (\underline{atleta}, \underline{gioco}, \underline{edizione}, posizione) \\\\
	\end{itemize}
	
	
	\section{Normalizzazione}
	
	Riguardo la normalizzazione si ha che:
	\begin{itemize}
		\item la prima forma normale è rispettata dato che tutte le colonne sono atomiche
		\item la seconda forma normale è rispettata dato che la prima forma normale è rispettata e ciascuna colonna dipende dalla chiave primaria
		\item la terza forma normale è rispettata dato che la seconda forma normale è rispettata ed ogni attributo dipende solo dalla chiave primaria\\\\
	\end{itemize}
	
	Una spiegazione leggermente più dettagliata del perché la base di dati è in seconda forma normale è data dal fatto che, riguardo la tabella "gioco", questa presenta una chiave primaria composta dagli attributi "disciplina" ed "edizione"; l' attributo "descrizione" dipende dall' attributo "disciplina", in cui si ha una descrizione del gioco ma dipende anche dall' attributo "edizione" dato che può succedere che un gioco, nel corso delle edizioni, può cambiare regole, modalità e quindi a sua volta la descrizione stessa.\\\\
	Osserviamo invece che la tabella "competizione" ha anch' essa una chiave primaria composta dagli attributi "atleta", "gioco" ed "edizione" e dato che l' atleta può competere sia in più giochi di una stessa edizione che in più edizioni, l' attributo "posizione", che rappresenta la posizone di un atleta nel gioco in cui partecipa, dipende da tutte le componenti della chiave primaria.
	
	\newpage
	
	\lstset{language=SQL}
	\begin{lstlisting}
		
		
		
		
	\end{lstlisting}
	
	\newpage
	
	\section{Query Aggiuntive}
	
	\lstset{language=SQL}
	\begin{lstlisting}[deletekeywords={IDENTITY,INT},
		morekeywords={clustered},    
		framesep=10pt,
		framextopmargin=10pt]
		--SP per ottenere il numero di biglietti rimanenti degli stadi di una data edizione
		
		delimiter &&
		create procedure getBigliettiDisponibili(in edizioneInput varchar(8))
		begin
		select s.nome as stadio, capienza - count(*) as biglietti_disponibili
		from biglietto
		inner join stadio s
		on biglietto.stadio = s.nome
		inner join zona z
		on s.zona = z.nome
		inner join citta c
		on z.citta = c.nome
		inner join olimpiade o
		on c.nome = o.sede
		where edizione like edizioneInput
		group by s.nome;
		end &&
		delimiter ;
		
		
		--SP per ottenere la classifica delle nazioni con piu atleti sul podio in una data edizione
		
		delimiter $$
		create procedure getMedagliere(in edizioneInput varchar(8))
		begin
		select nazione, count(*) as numMedaglie
		from atleta a
		inner join competizione c
		on a.ID = c.atleta
		inner join gioco g
		on ( c.gioco = g.disciplina 
		and c.edizione = g.edizione )
		where ((posizione = 1 or posizione = 2 or posizione = 3)
		and g.edizione like edizioneInput )
		group by nazione
		order by numMedaglie desc;
		end $$
		delimiter ;
		
		
		--SP per ottenere gli investimenti ed il numero di stadi messi a disposizione da ogni citta per ogni edizione
		
		delimiter $$
		create procedure getInvestimenti()
		begin
		select edizione, c.nome as citta, investimento, 
		count(s.nome) as numeroStadi
		from olimpiade o
		inner join citta c on o.sede = c.nome
		inner join zona z on c.nome = z.citta
		inner join stadio s on z.nome = s.zona
		group by edizione, c.nome, investimento
		order by edizione;
		end $$
		delimiter ;
		
		
		--SP per ottenere gli atleti piu premiati (sul podio) in tutte le edizioni. Utilizzo una User Defined Function per evitare di ripetere righe di codice.
		
		delimiter $$
		create function getNumMedaglie(id varchar(8), posizioneInput int(5))
		returns int(5)
		deterministic
		begin
		declare numeroMedaglie int(5);
		select count(*) into numeroMedaglie 
		from competizione
		where atleta = id and posizione = posizioneInput;
		return numeroMedaglie;
		end $$
		delimiter ;
		
		
		delimiter $$
		create procedure getAtletiPiuPremiati()
		begin
		select concat(a.nome, ' ', a.cognome) as nome,
		getNumMedaglie(a.ID, 1)        as numeroOri,
		getNumMedaglie(a.ID, 2)        as numeroArgenti,
		getNumMedaglie(a.ID, 3)        as numeroBronzi
		from atleta a
		inner join competizione c on a.ID = c.atleta
		where posizione = 1
		or posizione = 2
		or posizione = 3
		group by ID
		order by numeroOri desc, numeroArgenti desc, numeroBronzi desc;
		end $$
		delimiter ;
		
		
		--Trigger per controllare che non vengano venduti piu biglietti della capienza dello stadio
		
		delimiter $$
		create trigger checkCapienza
		before insert
		on biglietto
		for each row
		begin
		declare capienzaStadio int(11);
		declare numeroBiglietti int(11);
		select capienza into capienzaStadio 
		from stadio
		where nome like NEW.stadio;
		select count(*) into numeroBiglietti 
		from biglietto 
		where stadio like NEW.stadio;
		if (numeroBiglietti >= capienzaStadio) then
		set @signal = concat('Capienza piena per lo stadio ', NEW.stadio);
		SIGNAL SQLSTATE '45001' SET MESSAGE_TEXT = @signal;
		end if;
		end $$
		delimiter ;
		
		
		--Trigger per controllare che il numero della postazione di un biglietto non sia maggiore della capienza dello stadio
		
		delimiter $$
		create trigger checkPostazione
		before insert
		on biglietto
		for each row
		begin
		declare capienzaStadio int(11);
		select capienza into capienzaStadio 
		from stadio 
		where nome like NEW.stadio;
		if (NEW.postazione > capienzaStadio) then
		set @signal = concat('La postazione ', NEW.postazione, 
		' per lo stadio ', NEW.stadio, ' non e valida');
		SIGNAL SQLSTATE '45001' SET MESSAGE_TEXT = @signal;
		end if;
		end $$
		delimiter ;
		
		
		--Trigger per controllare che un atleta abbia almeno 18 anni rispetto alla data di inizio dell edizione in cui compete
		
		delimiter $$
		create trigger checkAge
		before insert
		on competizione
		for each row
		begin
		declare dataNascitaAtleta date;
		declare dataEdizione date;
		select dataNascita into dataNascitaAtleta 
		from atleta 
		where id = NEW.atleta;
		select dataApertura
		into dataEdizione
		from cerimonia c
		inner join olimpiade o on c.codice = o.cerimonia
		where o.edizione like NEW.edizione;
		if (year(dataEdizione) - year(dataNascitaAtleta) < 18) then
		set @signal =
		concat('L atleta ', (select concat(a.nome, ' ', a.cognome) 
		from atleta a 
		where id like NEW.atleta),
		' e troppo giovane per competere nell edizione ',
		(select edizione 
		from olimpiade o 
		where o.edizione like NEW.edizione), '. Tra ',
		(18 - (year(sysdate()) - year(dataNascitaAtleta))),
		' anni potra partecipare alla prossima edizione');
		SIGNAL SQLSTATE '45001' SET MESSAGE_TEXT = @signal;
		end if;
		end $$
		delimiter ;
		
		
		--Trigger per controllare che una postazione di un biglietto per uno stadio non sia duplicata
		
		delimiter $$
		create trigger checkPostazioniDuplicate
		before insert
		on biglietto
		for each row
		begin
		set @postazione = -1;
		select postazione into @postazione 
		from biglietto 
		where postazione = NEW.postazione and stadio = NEW.stadio;
		if (@postazione > 0) then
		set @signal = concat('La postazione ', NEW.postazione, 
		' per lo stadio ', NEW.stadio, 
		' e gia occupata. Trovare un altra postazione!');
		SIGNAL SQLSTATE '45001' SET MESSAGE_TEXT = @signal;
		end if;
		
		end $$
		delimiter ;
		
		
		-- Vista per ottenere gli atleti che hanno partecipato alle edizioni delle olimpiadi, la relativa nazione, disciplina e posizione di gara
		
		create view atletiPartecipanti as
		select concat(a.nome, ' ', a.cognome) as atleta, nazione, 
		disciplina, posizione, c.edizione
		from atleta a
		inner join competizione c on a.ID = c.atleta
		inner join gioco g on (c.gioco = g.disciplina 
		and c.edizione = g.edizione)
		order by edizione, disciplina, posizione;
		
		
		-- Crezione della tabella "spettatore" per soddisfare il vincolo non esprimibile graficamente in cui l eta di chi acquista un biglietto per un qualsiasi stadio deve essere almeno di 18 anni
		
		drop table if exists spettatore;
		
		create table spettatore (
		CF      varchar(20) not null,
		nome    varchar(20) not null,
		cognome varchar(20) not null,
		eta     int(11)     not null,
		primary key (CF),
		constraint checkSpettatoreAge check (eta >= 18)
		);
		
		
		-- Crezione della tabella "cerimonia" per soddisfare il vincolo non esprimibile graficamente in cui la data di apertura di un edizione di un olimpiade deve essere precedente alla data di chiusura della stessa edizione dell olimpiade
		
		drop table if exists cerimonia;
		
		create table cerimonia (
		codice       varchar(8) not null,
		dataApertura date       not null,
		dataChiusura date       not null,
		primary key (codice),
		constraint cerimonia_ibfk_1 check ( dataApertura < dataChiusura )
		);
		
		
		-- Crezione della tabella "competizione" per soddisfare il vincolo non esprimibile graficamente in cui il numero che rappresenta il posizionamento di un atleta deve essere positivo
		
		drop table if exists competizione;
		
		create table competizione (
		atleta    varchar(8)  not null,
		gioco     varchar(20) not null,
		edizione  varchar(8)  not null,
		posizione int(11)     not null,
		primary key (atleta, gioco, edizione),
		constraint competizione_ibfk_1 foreign key (atleta) references atleta (ID),
		constraint competizione_ibfk_2 foreign key (gioco, edizione) references gioco (disciplina, edizione),
		constraint competizione_ibfk_3 check ( posizione > 0 )
		);
	\end{lstlisting}
	
\end{document}